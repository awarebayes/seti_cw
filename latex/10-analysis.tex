\section{Аналитический раздел}
% выбор направления исследований, включающий обоснование направления исследования,
% методы решения задач и их сравнительную оценку
% описание выбранной общей методики проведения НИР;


Аналитический раздел проекта включает собой обоснование выбранного направления исследования, методы решения задач и их сравнительную оценку.


\subsection{Требования к серверу}

Можно привести следующие требования к разрабатываемому серверу:
\begin{enumerate}[label={\arabic*)}]
    \item сервер должен поддерживать GET, HEAD запросы;
    \item реализация HTTP статусов 200 (OK), 404 (NOT FOUND), 403 (FORBIDDEN), 405 (UNSUPPORTED METHOD);
    \item поддержка типов MIME, задаваемых в конфигурации;
    \item робастная работа веб-сервера (до 300 запросов в секунду для select, неограниченно запросов для epoll);
    \item корректная передача больших файлов;
    \item корректный парсинг URL с query, anchor, эскейп-последовательностями;
    \item сервер не должен иметь доступ к файлам вне рабочей директории.
\end{enumerate}

Автором были введены дополнительные требования в виде листинга директорий, поддержка статуса 304 (NOT CHANGED), поддержка заголовка Range.

\subsection{Протокол HTTP}

Рассмотрение современных веб-технологий невозможно без глубокого понимания протокола HTTP (Hypertext Transfer Protocol). Этот стандарт представляет собой основной механизм взаимодействия между веб-клиентами и серверами, обеспечивая передачу данных в формате гипертекста. HTTP сыграл ключевую роль в развитии интернета и является основой для передачи ресурсов, таких как HTML-страницы, изображения, стили и другие \cite{nets1}.

Протокол HTTP оперирует по принципу запрос-ответ, где клиент посылает запрос на сервер для получения определенного ресурса, и сервер возвращает соответствующий ответ. Этот обмен сообщениями строится вокруг простых принципов и стандартов, что позволяет веб-клиентам и серверам эффективно обмениваться информацией.

Существует несколько версий протокола HTTP, каждая из которых вносит улучшения и нововведения. В контексте данной работы выбран HTTP 1.1, поскольку он предоставляет возможность многократного использования соединений (connection reuse). Эта характеристика позволяет снизить задержки при обмене данными, поскольку после завершения одного запроса соединение может быть переиспользовано для последующих запросов, сэкономив трафик и улучшив производительность. В работе более детально рассмотрены механизмы переиспользования соединений в рамках HTTP 1.1 и их влияние на эффективность веб-взаимодействия.

В рамках рассмотрения альтернативных версий протокола HTTP, особое внимание привлекает HTTP/2.0. Эта версия была представлена с целью оптимизации производительности и улучшения пользовательского опыта в сравнении с ее предшественником, HTTP/1.1. Основными изменениями, внесенными в HTTP/2.0, стали многопоточность, мультиплексирование и сжатие заголовков, что привело к уменьшению задержек и более эффективному использованию ресурсов.

Мультиплексирование в HTTP/2.0 позволяет одновременно передавать несколько запросов и ответов в рамках одного соединения. Это снижает задержки и улучшает пропускную способность, что особенно важно в условиях медленных сетей или при использовании мобильных устройств. Сжатие заголовков также содействует более эффективной передаче данных \cite{nets2}.

Тем не менее, несмотря на ряд явных преимуществ, выбор HTTP/2.0 для данного проекта не был сделан. HTTP/2.0, несмотря на свою эффективность, обладает более сложной структурой и требует более глубокого понимания принципов мультиплексирования и других инноваций, что может создать дополнительные сложности в учебных целях. 

Таким образом, выбор HTTP 1.1 для данного проекта обоснован стремлением к более непосредственному и понятному представлению студентам основ работы веб-сервера и протокола HTTP.

\subsection{Сокеты UNIX}

Сетевые сокеты представляют собой универсальный механизм для обмена данными между процессами на одной машине или по сети. В контексте разработки веб-сервера, использование сетевых сокетов становится неотъемлемой частью, обеспечивая взаимодействие между сервером и клиентами.

Когда речь заходит о сетевых сокетах, невозможно не упомянуть дуплексную связь, которая предоставляет возможность полнодуплексного обмена данными между сервером и клиентом. Дуплексная связь становится ключевым аспектом в контексте веб-сервера, позволяя обрабатывать как входящие, так и исходящие запросы одновременно, что, в свою очередь, способствует повышению производительности и отзывчивости сервера.

Файловые сокеты, также известные как доменные сокеты UNIX, играют важную роль в сетевой коммуникации на локальном уровне. Они предоставляют способ для процессов обмениваться данными непосредственно, обеспечивая таким образом эффективную и безопасную передачу информации между частями приложения.

Выбор сетевых сокетов для реализации данного веб-сервера был обусловлен их универсальностью, способностью обработки запросов через сеть и локально. Сетевые сокеты обеспечивают надежное взаимодействие между сервером и клиентами, что существенно для корректной работы веб-сервера \cite{nets1}.

\subsection{Вывод}
На основе полученных данных были сформированы требования к статическому веб серверу. Обоснован выбор протокола HTTP 1.1 из-за эффективности переипользования соединений. Был обоснован выбор сетевых сокетов UNIX вместо файловых.