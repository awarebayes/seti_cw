\section*{\hfill{\centering ВВЕДЕНИЕ }\hfill}
\addcontentsline{toc}{section}{ВВЕДЕНИЕ}

Целью представленного проекта является разработка статического веб-сервера, способного обрабатывать GET и HEAD запросы. Статический веб-сервер представляет из себя программное решение, нацеленное на обслуживание и доставку статического контента через сеть Интернет. В его обязанности входит предоставление доступа к разнообразным ресурсам, таким как HTML-страницы, изображения, таблицы стилей, и прочее, которые не подлежат динамической генерации на стороне сервера.

В связи с эволюцией веб-технологий статические веб-серверы занимают особое положение, обеспечивая надежную и быструю обработку статических ресурсов. Их востребованность обусловлена простотой и эффективностью в обеспечении пользователей быстрым доступом к статическим компонентам веб-приложений.

Разработка статических веб-серверов сталкивается с проблемами безопасности, такими как возможность выхода из предварительно заданной директории. Потенциальные угрозы данного рода могут привести к компрометации конфиденциальных данных и иным нежелательным последствиям для сервера. В данной работе будет уделено внимание методам обеспечения безопасности статического веб-сервера, включая меры контроля доступа и предотвращения уязвимостей, связанных с управлением файловой системой сервера.

Для достижения поставленной цели, предполагается выполнение следующих задач:
\begin{enumerate}[label={\arabic*)}]
    \item провести формализацию задачи и определить необходимый функционал;
    \item исследовать предметную область веб серверов;
    \item спроектировать приложение;
    \item реализовать приложение;
    \item протестировать приложение на предмет корректности;
\end{enumerate}